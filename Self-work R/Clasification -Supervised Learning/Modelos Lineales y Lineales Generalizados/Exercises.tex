\documentclass[english,]{article}
\usepackage{lmodern}
\usepackage{amssymb,amsmath}
\usepackage{ifxetex,ifluatex}
\usepackage{fixltx2e} % provides \textsubscript
\ifnum 0\ifxetex 1\fi\ifluatex 1\fi=0 % if pdftex
  \usepackage[T1]{fontenc}
  \usepackage[utf8]{inputenc}
\else % if luatex or xelatex
  \ifxetex
    \usepackage{mathspec}
  \else
    \usepackage{fontspec}
  \fi
  \defaultfontfeatures{Ligatures=TeX,Scale=MatchLowercase}
\fi
% use upquote if available, for straight quotes in verbatim environments
\IfFileExists{upquote.sty}{\usepackage{upquote}}{}
% use microtype if available
\IfFileExists{microtype.sty}{%
\usepackage{microtype}
\UseMicrotypeSet[protrusion]{basicmath} % disable protrusion for tt fonts
}{}
\usepackage[margin=1in]{geometry}
\usepackage{hyperref}
\hypersetup{unicode=true,
            pdftitle={Exercises},
            pdfauthor={Manuel Gijón Agudo},
            pdfborder={0 0 0},
            breaklinks=true}
\urlstyle{same}  % don't use monospace font for urls
\ifnum 0\ifxetex 1\fi\ifluatex 1\fi=0 % if pdftex
  \usepackage[shorthands=off,main=english]{babel}
\else
  \usepackage{polyglossia}
  \setmainlanguage[]{english}
\fi
\usepackage{graphicx,grffile}
\makeatletter
\def\maxwidth{\ifdim\Gin@nat@width>\linewidth\linewidth\else\Gin@nat@width\fi}
\def\maxheight{\ifdim\Gin@nat@height>\textheight\textheight\else\Gin@nat@height\fi}
\makeatother
% Scale images if necessary, so that they will not overflow the page
% margins by default, and it is still possible to overwrite the defaults
% using explicit options in \includegraphics[width, height, ...]{}
\setkeys{Gin}{width=\maxwidth,height=\maxheight,keepaspectratio}
\IfFileExists{parskip.sty}{%
\usepackage{parskip}
}{% else
\setlength{\parindent}{0pt}
\setlength{\parskip}{6pt plus 2pt minus 1pt}
}
\setlength{\emergencystretch}{3em}  % prevent overfull lines
\providecommand{\tightlist}{%
  \setlength{\itemsep}{0pt}\setlength{\parskip}{0pt}}
\setcounter{secnumdepth}{0}
% Redefines (sub)paragraphs to behave more like sections
\ifx\paragraph\undefined\else
\let\oldparagraph\paragraph
\renewcommand{\paragraph}[1]{\oldparagraph{#1}\mbox{}}
\fi
\ifx\subparagraph\undefined\else
\let\oldsubparagraph\subparagraph
\renewcommand{\subparagraph}[1]{\oldsubparagraph{#1}\mbox{}}
\fi

%%% Use protect on footnotes to avoid problems with footnotes in titles
\let\rmarkdownfootnote\footnote%
\def\footnote{\protect\rmarkdownfootnote}

%%% Change title format to be more compact
\usepackage{titling}

% Create subtitle command for use in maketitle
\newcommand{\subtitle}[1]{
  \posttitle{
    \begin{center}\large#1\end{center}
    }
}

\setlength{\droptitle}{-2em}

  \title{Exercises}
    \pretitle{\vspace{\droptitle}\centering\huge}
  \posttitle{\par}
    \author{Manuel Gijón Agudo}
    \preauthor{\centering\large\emph}
  \postauthor{\par}
      \predate{\centering\large\emph}
  \postdate{\par}
    \date{8/12/2018}


\begin{document}
\maketitle

{
\setcounter{tocdepth}{2}
\tableofcontents
}
\section{Linear Models}\label{linear-models}

\subsection{Linear Regression}\label{linear-regression}

\subsubsection{Exercise 1}\label{exercise-1}

It is well known that the excess of weight is one of the factors that
has a negative influence in the cholesterol level of human beings. In an
experiment with children from 9 to 20 years old, measures of their
cholesterol level (C), the weight (W) the height (H) and the age (A)
have been recorded. The data appear in file COL.csv. 1. Compute the
regression line for modeling the cholesterol as a function of weight. 2.
Plot the regression line jointly with the confidence intervals and the
prediction intervals. 3. Perform the appropriate plots to check: 1.
Tendency and homogeneity of variances. Plot the residuals as a function
of the predicted values. 2. Outliers. Plot the studentized residuals as
a function of any of the predictors, observation number \ldots{} jointly
with the horizontal lines at −2 and 2. 3. Influence values. Plot the
dffits as a function of any of the following variables: predicted
values, observation number \ldots{}; jointly with the horizontal lines
at \(2 \sqrt(\frac{p}{n})\)

\subsubsection{Exercise 2}\label{exercise-2}

\subsubsection{Exercise 3}\label{exercise-3}

\subsubsection{Exercise 4}\label{exercise-4}

\subsubsection{Exercise 5}\label{exercise-5}

\subsection{One Way Anova}\label{one-way-anova}

\subsubsection{Exercise 6}\label{exercise-6}

\subsubsection{Exercise 7}\label{exercise-7}

\subsubsection{Exercise 8}\label{exercise-8}

\subsection{Two way and more than Two way
ANOVA}\label{two-way-and-more-than-two-way-anova}

\subsubsection{Exercise 9}\label{exercise-9}

\subsubsection{Exercise 10}\label{exercise-10}

\subsubsection{Exercise 11}\label{exercise-11}

\subsubsection{Exercise 12}\label{exercise-12}

\subsubsection{Exercise 13}\label{exercise-13}

\subsection{ANCOVA: Analysis of
covariance}\label{ancova-analysis-of-covariance}

\subsubsection{Exercise 14}\label{exercise-14}

\subsubsection{Exercise 15}\label{exercise-15}

\subsubsection{Exercise 16}\label{exercise-16}

\section{Generalized Linear Models}\label{generalized-linear-models}

\subsection{General formulation}\label{general-formulation}

\subsubsection{Exercise 17}\label{exercise-17}

\subsection{Continuous models}\label{continuous-models}

\subsubsection{Exercise 18}\label{exercise-18}

\subsubsection{Exercise 19}\label{exercise-19}

\subsubsection{Exercise 20}\label{exercise-20}

\subsubsection{Exercise 21}\label{exercise-21}

\subsection{Binomial and Poisson
models}\label{binomial-and-poisson-models}

\subsubsection{Exercise 22}\label{exercise-22}

\subsubsection{Exercise 23}\label{exercise-23}

\subsubsection{Exercise 24}\label{exercise-24}

\subsubsection{Exercise 25}\label{exercise-25}

\subsubsection{Exercise 26}\label{exercise-26}

\subsection{Advanced exercises}\label{advanced-exercises}

\subsubsection{Exercise 27}\label{exercise-27}

\subsubsection{Exercise 28}\label{exercise-28}


\end{document}
